%------------------------------------------------------------
%	PREÁMBULO
%------------------------------------------------------------
\documentclass[11pt, twocolumn]{article} 

\usepackage{graphics,graphicx}
\usepackage{multicol}
\usepackage{multirow}
\usepackage{fancyhdr}
\usepackage{enumerate}
\usepackage[spanish,es-nodecimaldot,es-tabla]{babel}
\usepackage[utf8]{inputenc}
\usepackage[title]{appendix}
\usepackage{url}
\usepackage[hidelinks]{hyperref}
\usepackage{braket}
\usepackage{caption}
\usepackage{subcaption}
\usepackage{afterpage}
\usepackage{titling} 
\usepackage{float}
\usepackage{lscape}
\usepackage{selinput}
\usepackage{booktabs} 
\usepackage{lettrine}
\usepackage{color}
\usepackage{cancel}
\usepackage{tikz}
\usepackage{lipsum}

\usepackage{amsfonts} 
\usepackage[centertags]{amsmath}
\usepackage{stmaryrd,amssymb,amsthm}
\usepackage{wasysym,mathrsfs}

\usepackage[font=footnotesize,labelfont=small]{caption}
\captionsetup{width=0.85\linewidth}

\RequirePackage{geometry}
\geometry{margin=1.5cm}

\usepackage{parskip}
\setlength{\parskip}{0.2cm}
\setlength{\parindent}{0pt}

\usepackage[square,numbers,sort]{natbib}
\bibliographystyle{unsrt}

\selectlanguage{spanish}

%------------------------------------------------------------
%	DEFINICIONES
%------------------------------------------------------------
\def\path{Images/}

\newcommand{\sol}{
	\begin{flushleft}
		$ \mathbb{S}ol. $
\end{flushleft}}


%------------------------------------------------------------
%	CARÁTULA
%------------------------------------------------------------
\author{Fulano}
\title{Reporte}
\date{}

\begin{document} 
	
\twocolumn[\begin{@twocolumnfalse}
		
\begin{center}

	{\Large Título del experimento} \vspace{5pt}
		
	Fulanito \textsc{de Tal}\footnotemark \\
	Facultad de Ciencias, \textsl{UNAM} \\
	Laboratorio de Física Contemporánea I: Proyecto Nº 1 \\
	Profesores: Dr. Perengano de Tal \& Ayudante Perenganito \vspace{5pt} 

	Diciembre, 2024
			
\end{center}


%------------------------------------------------------------
%	ABSTRACT
%------------------------------------------------------------

\rule{\textwidth}{1pt}

\begin{abstract}
	En el resumen se evalúa la capacidad de síntesis del estudiante. Es importante que en muy pocas palabras redacte el por qué es importante el experimento, lo que realizó, los resultados y comentarios pertinentes, como una justificación de las fallas cometidas en la experimentación. Se debe escribir en pasado perfecto.
\end{abstract}

\rule{\textwidth}{1pt}

\vspace{20pt}

\end{@twocolumnfalse}]

\footnotetext{\url{correo_de_fulanito@ciencias.unam.mx}}


%------------------------------------------------------------
%	INTRODUCCIÓN
%------------------------------------------------------------
\section{Introducción}
\label{sec:Intro}

Los reportes experimentales son de gran importancia al momento de cursar un laboratorio ya que no solo permiten a los y las alumnas desarrollar habilidades de comunicación escrita; sino porque describen los procesos, datos y conocimientos que recabó el o la alumna a lo largo del proceso experimental.

Para la introducción no se espera que se desarrolle toda la teoría física que sustente el experimento, pero sí que aquí se encuentren bases sólidas. 

Es opcional dar un breve esbozo histórico del experimento para dar contexto del mismo y no debe ser el contenido central de la introducción.


%------------------------------------------------------------
\subsection{Ecuaciones}
\label{subsec:Ecuaciones}

En cuando a ecuaciones -en caso de necesitarlas- no es necesario deducirlas todas, pero sí explicar su significado físico en el ambiente en el que se esté trabajando.

%------------------------------------------------------------
\subsection{Objetivos}
\label{subsec:Objetivos}

Siempre es preferible hacer una referencia directa a las imágenes que se agregan, como en la Figura \ref{fig:Gatito}.

\begin{figure}[H]
	\centering
	\includegraphics[width=0.7\columnwidth]{\path yolo.jpg}
	\caption{Un ejemplo de que los gatitos siempre muestran su cariño.}
	\label{fig:Gatito}
\end{figure}

Los objetivos deben de ser claros desde el comienzo de la experimentación y esa claridad debe verse reflejada en esta sección. Así mismo deben de ser concisos y estar escritos en infinitivo.

No se recomienda que estén escritos en forma de lista, sino agregados de forma orgánica.


%------------------------------------------------------------
%	DESARROLLO
%------------------------------------------------------------
\section*{Desarrollo experimental}
\label{sec:Desarrollo}

En esta sección se describirá el arreglo experimental y las actividades que se realizaron durante la actividad. Pueden complementar la información el uso de diagramas.

Se debe escribir explícitamente que los diagramas extraídos de alguna fuente pertenecen a dicha fuente, como en la Figura \ref{fig:Geons}, extraída de \cite{wheeler1955}.

\begin{figure}[H]
	\centering
	\includegraphics[width=1\columnwidth]{\path geons.png}
	\caption{El geon más simple que se puede describir. Diagrama extraído de \cite{wheeler1955}.}
	\label{fig:Geons}
\end{figure}

Se sugiere el uso de diagramas en vez de fotografías debido a que visualmente es más atractivo, más limpio y más fácil de leer. Cada figura debe tener pie de imagen que describa lo que se espera ver en el esquema. Estos pueden hacerse en algún \textit{software} de diseño gráfico o en alguna página especializada como \href{https://www.mathcha.io/}{\textit{mathcha}}.


%------------------------------------------------------------
%	RESULTADOS
%------------------------------------------------------------
\section{Resultados}
\label{sec:Resultados}

En la parte de resultados, al usar gráficas estas deben ser legibles. Eviten el uso de tablas. Y recuerden el uso de incertidumbres asociadas a mediciones y aparatos.

Por ejemplo, en la Figura \ref{fig:Hg} se graficó el espectro de emisión del mercurio en lugar de mostrar una tabla que muestre la intensidad relativa en función de la longitud de onda. Por la naturaleza de esta gráfica no es necesario agregar barras de error.

\begin{figure}[H]
	\centering
	\includegraphics[width=1\columnwidth]{\path hg.pdf}
	\caption{Espectro de emisión del mercurio, los datos fueron extraídos de \cite{nistHg}.}
	\label{fig:Hg}
\end{figure}

En la Figura \ref{fig:Capacitancia} se muestran datos en los que sí es necesario agregar barras de error. 

\begin{figure}[H]
	\centering
	\includegraphics[width=1\columnwidth]{\path capacitancia.pdf}
	\caption{Capacitancia en función de la separación entre sus placas. Se muestra un ajuste de la forma $ C = \frac{A}{d} + B $, donde $ C $ es la capacitancia, $ d $ es la distancia entre placas y por otro lado $ A $ y $ B $ son constantes a determinar.}
	\label{fig:Capacitancia}
\end{figure}

Aquí no se muestra el valor de las constantes $ A $ y $ B $ del ajuste porque solo se trata de un ejemplo, pero en un reporte usual sí deben de agregarse estos datos. Pero sí se muestra las unidades entre corchetes en los ejes. 

Además es importante que las citas se hagan por orden de aparición, primero se citó \cite{wheeler1955}, después \cite{nistHg} y para obtener la forma del ajuste anterior se usó \cite{griffithsElectrodynamics}. O sea que están en orden. Esto se hace en automático con la forma gestionar la bibliografía de este \textit{template}.


%------------------------------------------------------------
%	ANÁLISIS Y DISCUSIÓN
%------------------------------------------------------------
\section*{Análisis y discusión}
\label{sec:Analisis}

Se infiere que este es el bloque más importante del reporte dado que se trabajan las habilidades de análisis y comprensión de resultados.

Por ejemplo, el ajuste del que se habla en la Figura \ref{fig:Capacitancia} arroja el valor numérico de $ A $ y $ B $, con lo cual se infiere información física, por ejemplo $ A $ está relacionada con el área de las placas y la permitividad del vacío. 

Aquí es donde se unen las mediciones realizadas y la teoría que sustenta el experimento.


%------------------------------------------------------------
%	CONCLUSIONES
%------------------------------------------------------------
\section{Conclusiones}
\label{sec:Conclusión}

En esta sección debe explicar si los objetivos de la práctica se cumplieron o no, en caso de haber notado fallas experimentales se pueden sugerir mejoras. Es importante señalar que no es necesariamente malo tener resultados alejados a la teoría, en caso de ser el caso es de suma importancia señalar qué causó este resultado y la mejora recomendada.


%------------------------------------------------------------
%	APÉNDICES
%------------------------------------------------------------
\appendix
\section*{Apéndice}
\label{sec:Apéndice}

En este tipo de reportes es muy raro que se necesiten apéndices, pero en caso de necesitar uno o dos se pueden agregar. Aquí solo se agrega un \textit{lorem ipsum}.

\lipsum[1-2]

\newpage


%------------------------------------------------------------
%	BIBLIOGRAFÍA
%------------------------------------------------------------
\bibliography{./Bibliography/bibliography.bib}


\end{document}