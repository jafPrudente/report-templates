\section*{Desarrollo experimental}
\label{sec:Desarrollo}

En esta sección se describirá el arreglo experimental y las actividades que se realizaron durante la actividad. Pueden complementar la información el uso de diagramas.

Se debe escribir explícitamente que los diagramas extraídos de alguna fuente pertenecen a dicha fuente, como en la Figura \ref{fig:Geons}, extraída de \cite{wheeler1955}.

\begin{figure}[H]
	\centering
	\includegraphics[width=1\columnwidth]{\path geons.png}
	\caption{El geon más simple que se puede describir. Diagrama extraído de \cite{wheeler1955}.}
	\label{fig:Geons}
\end{figure}

Se sugiere el uso de diagramas en vez de fotografías debido a que visualmente es más atractivo, más limpio y más fácil de leer. Cada figura debe tener pie de imagen que describa lo que se espera ver en el esquema. Estos pueden hacerse en algún \textit{software} de diseño gráfico o en alguna página especializada como \href{https://www.mathcha.io/}{\textit{mathcha}}.