\section*{Introducción}
\label{sec:Intro}

Los reportes experimentales son de gran importancia al momento de cursar un laboratorio ya que no solo permiten a los y las alumnas desarrollar habilidades de comunicación escrita; sino porque describen los procesos, datos y conocimientos que recabó el o la alumna a lo largo del proceso experimental.

Para la introducción no se espera que se desarrolle toda la teoría física que sustente el experimento, pero sí que aquí se encuentren bases sólidas. 

Es opcional dar un breve esbozo histórico del experimento para dar contexto del mismo y no debe ser el contenido central de la introducción.


%------------------------------------------------------------
\subsection*{Ecuaciones}
\label{subsec:Ecuaciones}

En cuando a ecuaciones -en caso de necesitarlas- no es necesario deducirlas todas, pero sí explicar su significado físico en el ambiente en el que se esté trabajando.

%------------------------------------------------------------
\subsection*{Objetivos}
\label{subsec:Objetivos}

Siempre es preferible hacer una referencia directa a las imágenes que se agregan, como en la Figura \ref{fig:Gatito}.

\begin{figure}[H]
	\centering
	\includegraphics[width=0.7\columnwidth]{\path yolo.jpg}
	\caption{Un ejemplo de que los gatitos siempre muestran su cariño.}
	\label{fig:Gatito}
\end{figure}

Los objetivos deben de ser claros desde el comienzo de la experimentación y esa claridad debe verse reflejada en esta sección. Así mismo deben de ser concisos y estar escritos en infinitivo.

No se recomienda que estén escritos en forma de lista, sino agregados de forma orgánica.